\documentclass[12pt]{article}



\usepackage{graphicx}
\graphicspath{{Images/}}

\usepackage{hyperref}
\hypersetup{colorlinks=true,citecolor=black,linkcolor=black,urlcolor=black}




\begin{document}
\title{\huge National Institute of Technology Raipur}
\begin{figure}
\centering
\includegraphics[scale=0.2]{NITRR.jpg}
\end{figure}
\author{\textit{Submitted By:- Harshita Upendra Sakhare}\\ \textbf{Roll.No:- 21111049}\\ \textit{Submitted To:- Prof.Saurabh Gupta}\\ \textsc{Assignment-1}\\ \textsc{ON}\\ \textbf{Summary On An Introduction To Human Body}}

\maketitle
\clearpage
\tableofcontents
\clearpage

\section{Acknowledgement}
\hspace{1cm}
In successful completion of my assignment on \textit{ SUMMARY ON AN INTRODUCTION TO HUMAN BODY}, I would like to thank my
Professor. Saurabh Gupta Lecturer of Biomedical Engineering, who
has guided and assisted me to complete the assignment. Without
his support I would not have finished the assignment within time.
I would also like to take this opportunity to thank my friends
and family members,\\without them this assignment could not have
been completed in a short duration.
\clearpage

\section{Anatomy and Physiology Defined}
\hspace{1cm}
Anatomy is the science of the body structures and relationships among structures; physiology is the science of body functions. Dissection is the careful cutting apart of the body structures to study their relationship. Some branches of anatomy are embryology, developmental biology, cell biology, histology, gross anatomy, systemic biology, regional anatomy, surface anatomy, radiographic anatomy, and pathological anatomy.


\section{Levels of Structural Organization}
\hspace{1cm}
The human body consists of six levels of structural organization:chemical, cellular, tissue, organ, system and organism. Cells are the basic structural and functional living units of an organism and are the smallest living units in the human body. Tissues are group of cells and the materials surrounding them that work together to perform particular function. Organs are composed of two or more different types of tissues;they have specific functions and usually have recognizable shapes. Systems consists of related organs that have a common function. An organism is any living individual.

\section{Characteristics of the Living Human Organism}
\hspace{1cm}
All organism carry on certain processes that distinguish them from non-living things. Among the life processes in human are metabolism, responsiveness, movement, growth, differentiation, and reproduction.

\section{Homeostasis}
\hspace{1cm}
Homeostasis is the maintenance of relatively stable conditions in the body's internal environment produced by the interplay of all of the body's regulatory processes. Body fluids are dilutes, watery solutions. Intracellular fluid (ICF) is inside cells, and extracellular fluid (ECF) is outside cells. Plasma is the ECF within blood vessels. Interstitial fluid is the ECF that fills spaces between tissue cells. Because it surrounds the cells of the body, Extracellular fluid is also called the body's internal environment. Disruptions of homeostasis come from external and internal stimuli and psychological stresses. When disruption of homeostasis is mild and temporary, responses of body cells quickly restore balance in the internal environment. If disruption is extreme, regulation of homeostasis may fail.

\section{Basic Anatomical Terminology}
\hspace{1cm}
Descriptions of any region of the body assume the body is in the anatomical position, in which the subject stands erect facing the observer, with the head level and the eyes facing directly forward. The feet are flat on the floor and directed forward, and the upper limbs are at the sides, with the palms, turned forward. A body lying face down is prone; a body lying face up is supine. Regional names are terms given to specific regions of the body. The principal regions are the head, neck, trunk, upper limbs and lower limbs. Within the regions, specific body parts have anatomical names and corresponding common names. Examples are thoracic(chest), nasal(nose) and carpal(wrist). Directional terms indicate the relationship of one part of the body to another. 

\section{Aging and Homeostasis}
\hspace{1cm}
Aging produces observable changes in structure and function and increases vulnerability to stress and disease. Changes associated with aging occur in all body systems.

\section{Medical Imaging}
\hspace{1cm}
Medical Imaging refers to techniques and procedures used to create images of the human  body. They allows visualization of internal structures to diagnose abnormal anatomy and devices from normal physiology.
\clearpage


\section{Clinical Connections}
\subsection{Noninvasive Diagnostic Techniques}
\hspace{1cm}
Heath-care professionals and students of anatomy and physiology commonly use several noninvasive techniques to assess certain aspects of body structure and function. A noninvasive  diagnostic technique is one that does not involve insertion of an instrument or device through the skin or a body opening. In inspection the examiner observes the body for any changes that deviate from normal. In palpation the examiner feels the body surfaces with the hands. In auscultation the examiner listen to the body sounds to evaluate the functioning of certain organs, often using a stethoscope to amplify the sounds. In percussion the examiner taps on the body surface with the fingertips and listens to the resulting sound. Hollow cavities or spaces produce a different sound than solid organs.

\subsection{Autopsy}
\hspace{1cm}
An autopsy or necropsy is a post-mortem examination of the body and discussion of its internal organs to confirm or determine the cause of death. An autopsy can uncover the existence of diseases not detected during life, determine the extent of injuries, and explain how those injuries may have contributed to a person's death. It also may provide more information about disease, assist in the accumulation of statistical data, and educate health-care students. Moreover, an autopsy can reveal conditions that may affect offspring or siblings.

\subsubsection{Diagnosis of Disease}
\hspace{1cm}
Diagnosis  is the science and skill of distinguishing one disorder or disease from another. The patient's symptoms and signs, his or her medical history, a physical exam, and laboratory tests provide the basis for making a diagnosis. Taking a medical history consists of collecting information about events that might be related to a patient's illness. These include chief complaint history of present illness, past medical problems, family medical problems , social history and review of symptoms. A physical examination is an orderly evaluation of the body and its functions. This process includes noninvasive techniques of inspection, palpation, auscultation, and percussion along with the measurement of vital signs and sometimes laboratory tests. 

 




\end{document}